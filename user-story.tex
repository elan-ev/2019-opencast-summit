\documentclass[a4paper]{article}

\usepackage[a4paper,twoside,bindingoffset=1cm,top=1in,bottom=1in,right=1.5in,left=1.5in]{geometry}% http://ctan.org/pkg/geometry
\usepackage{graphicx}
\usepackage[colorlinks=true, unicode=true]{hyperref}
\usepackage[utf8]{inputenc}
\usepackage[english]{babel}
\usepackage[T1]{fontenc}
\usepackage{booktabs}
\usepackage{microtype}
\usepackage{lmodern}
\usepackage{xspace}
\usepackage{textcomp}
\usepackage{parskip}

\hypersetup{colorlinks=true, linkcolor=black, urlcolor=black}

\usepackage{libertine}
\renewcommand*\familydefault{\sfdefault}

\urlstyle{same}

\title{Opencast as a Lecturer (User Story)}
\date{Dec 12, 2018}
\author{Lars Kiesow (Osnabrück University)}

\begin{document}

\maketitle

\begin{center}
\begin{tabular}{ll}
\toprule
	Type            & Lightning Talk (15min) \\
	Target Audience & Users, Beginners \\
\bottomrule
\end{tabular}
\end{center}

\vspace{1em}

This lightning talk showcases the usage of Opencast seen from the perspective
of a lecturer at Osnabrück University.

In 2018, I worked at Osnabrück University as a lecturer where I used Opencast
to record a seminar about “Open-Source Software Development” in which we used
Opencast to record all sessions: both mine and the student's sessions. This
allowed them to review their own performances and provided content for other
students to easily reference in successive sessions.

I will shortly present my experiences when using Opencast. What did work out
well? What didn't? What are things you should be aware of? How did the students
react?

\end{document}
