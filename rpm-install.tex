\documentclass[a4paper]{article}

\usepackage[a4paper,twoside,bindingoffset=1cm,top=1in,bottom=1in,right=1.5in,left=1.5in]{geometry}% http://ctan.org/pkg/geometry
\usepackage{graphicx}
\usepackage[colorlinks=true, unicode=true]{hyperref}
\usepackage[utf8]{inputenc}
\usepackage[english]{babel}
\usepackage[T1]{fontenc}
\usepackage{booktabs}
\usepackage{microtype}
\usepackage{lmodern}
\usepackage{xspace}
\usepackage{textcomp}
\usepackage{parskip}

\hypersetup{colorlinks=true, linkcolor=black, urlcolor=black}

\usepackage{libertine}
\renewcommand*\familydefault{\sfdefault}

\urlstyle{same}

\title{Installation via Package Repository}
\date{Dec 12, 2018}
\author{Lars Kiesow (ELAN e.V.)}

\begin{document}

\maketitle

\begin{center}
\begin{tabular}{ll}
\toprule
	Type            & Lightning Talk (15min) \\
	Target Audience & Beginners \\
\bottomrule
\end{tabular}
\end{center}

\vspace{1em}

This lightning talk will show how installing Opencast from the official package
repository works and how you can get a working Opencast set-up in just a few
minutes.

This sessions is targeted toward potentially new Opencast adopters with no
prior knowledge about the installation process of Opencast. It will focus on
one way to archive this goal (RPM packages on CentOS) and will not get into
details about distributed or even custom set-ups.

\end{document}
