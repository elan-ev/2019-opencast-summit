\documentclass[a4paper]{article}

\usepackage[a4paper,twoside,bindingoffset=1cm,top=1in,bottom=1in,right=1.5in,left=1.5in]{geometry}% http://ctan.org/pkg/geometry
\usepackage{graphicx}
\usepackage[colorlinks=true, unicode=true]{hyperref}
\usepackage[utf8]{inputenc}
\usepackage[english]{babel}
\usepackage[T1]{fontenc}
\usepackage{booktabs}
\usepackage{microtype}
\usepackage{lmodern}
\usepackage{xspace}
\usepackage{textcomp}
\usepackage{parskip}

\hypersetup{colorlinks=true, linkcolor=black, urlcolor=black}

\usepackage{libertine}
\renewcommand*\familydefault{\sfdefault}

\urlstyle{same}

\title{Opencast Releases}
\date{Dec 12, 2018}
\author{Lars Kiesow (ELAN e.V.)}

\begin{document}

\maketitle

\begin{center}
\begin{tabular}{ll}
\toprule
	Type            & Talk (30min) \\
	Target Audience & Everyone \\
\bottomrule
\end{tabular}
\end{center}

\vspace{1em}

Last year, \href{https://elan-ev.de}{ELAN e.V.} and
\href{https://plapadoo.com}{plapadoo} initiated a successful crowdfunding
initiative for updating and improving some important components of Opencast.

We saw this as an important task for the future of Opencast as it helps us to
make some key components work with Java 9+ (something we could not avoid for
long) as well as improve the performance of Opencast's scheduler.

This talk strives to give an overview about how this crowdfunding initiative
came about and what the current status of the separate goals is. What has been
updated already? What has changed or or will change at all? How did we approach the tasks?


\section*{Goal 1: Update Apache Karaf}

\href{https://karaf.apache.org/}{Apache Karaf} is the OSGI library all of Opencast is based on. It
provides us with a set of common libraries we can use and which are
ensured to be secure, up-to-date and work together. To get these
updated libraries and, arguably more important, to support the new Java
9, we need an upgrade to \href{https://karaf.apache.org/download.html}{Karaf 4.2.x}.

If the first goal is reached, ELAN e.V. will update Apache Karaf to the
latest 4.2.x version. This includes updating the library chain, the
assembly plugin, and our distributions.


\section*{Goal 2: Update Elasticsearch}

Opencast is on
\href{https://www.elastic.co/products/elasticsearch}{Elasticsearch} version
1.7.6. Again, this does not
\href{https://www.elastic.co/support/matrix#matrix_jvm}{work with Java 9} and
is furthermore not supported anymore. This means it will not get any further
security updates or any other maintenance.

The Opencast community has already discussed in the past that upgrading
this is highly desirable and something that needs to be done in the
long run. This is not only for Java 9 compatibility but also as a
backend for an updated player/portal and to replace the old Solr
services. It was also one of the most discussed points in the recent
meetings of Opencast's new high-availability working group

The search indexes are what currently power (and limit!) all of our
user interfaces. In case of Elasticsearch, its inability to run
distributed currently makes it a single point of failure. That is a
problem when you think about highly available systems. To put it
bluntly: Upgrading Elasticsearch is one of the first steps we need to
do when thinking about removing the admin node as a single point of
failure.

The ELAN e.V. will update Elasticsearch to the latest 6.x version and
integrate it into the Opencast services which require it. This includes
the admin interface, external API and its companion, the index service.

We will keep in mind the requirements the for high-availability and
will ensure that Opencast will be able to support stand-alone versions
of Elasticsearch as well as shipping it's updated internal version for
smaller installations and ease of installation.


\section*{Goal 3: Scheduler Conflict Check Performance}

Opencast allows to schedule recordings using the Administrative User
interface or the External API. When new events are added, the system
has to check for conflicts since two events cannot be recorded at the
same time by the same capture agent.

Due to an architectural flaw, the checks performed by the system are
currently very slow. The more events and capture agents you have, the
more complex and slower this check gets. This leads to a slow,
unresponsive UI on the one hand and can bring the whole system down on
the other hand, in case someone creates many scheduled events within a
short period. This can happen by accident or normal usage and
represents a hidden risk for the availability of your Opencast
installation.

Changing the underlying data structure requires reworking a fair bit of
code. With scalability in mind, a new implementation will speed up the
conflict checking significantly. This will improve user experience and
availability on all sizes of Opencast installations.


plapadoo will rewrite/improve the scheduler to speed up conflict
checking by at least 50\% for a picked use case. Remaining funds will be
used to increase test coverage.

\end{document}
