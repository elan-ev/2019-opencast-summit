\documentclass[a4paper]{article}

\usepackage[a4paper,twoside,bindingoffset=1cm,top=1in,bottom=1in,right=1.5in,left=1.5in]{geometry}% http://ctan.org/pkg/geometry
\usepackage{graphicx}
\usepackage[colorlinks=true, unicode=true]{hyperref}
\usepackage[utf8]{inputenc}
\usepackage[english]{babel}
\usepackage[T1]{fontenc}
\usepackage{booktabs}
\usepackage{microtype}
\usepackage{lmodern}
\usepackage{xspace}
\usepackage{textcomp}
\usepackage{parskip}
\usepackage{xcolor}

\hypersetup{colorlinks=true, linkcolor=purple, urlcolor=purple}

\usepackage{libertine}
\renewcommand*\familydefault{\sfdefault}

\urlstyle{same}

\title{Crowdfunding}
\author{
	Michael Stypa (plapadoo) \\
	Lars Kiesow (ELAN e.V.)
}

\begin{document}

\maketitle

\begin{center}
\begin{tabular}{ll}
\toprule
	Type            & Talk (30min) \\
	Target Audience & Everyone \\
\bottomrule
\end{tabular}
\end{center}

\vspace{1em}

Last year, \href{https://elan-ev.de}{ELAN e.V.} and
\href{https://plapadoo.com}{plapadoo} initiated a successful crowdfunding
initiative for updating and improving core elements of Opencast.

We saw this as an important task for the future of Opencast as it helps us to
make some key components work with Java 9+ (something we could not avoid for
long anyway) as well as improve the performance and ensure scalability of
Opencast's scheduler.

This talk strives to give an overview about how this crowdfunding initiative
came about and what the current status of the separate goals is. What has been
updated already? What has changed or will change at all? How did we approach
the tasks?

This talk will not go into any technical details. If you are interested in
that, look out for related sessions in the technical track.


\section*{Update Apache Karaf}

\href{https://karaf.apache.org/}{Apache Karaf} is the OSGI library all of
Opencast is based on. It provides us with a set of common libraries we can use
and which are ensured to be secure, up-to-date and work together. To get these
updated libraries and, arguably more important, to support Java 9, we
need an upgrade to \href{https://karaf.apache.org/download.html}{Karaf 4.2.x}.


\section*{Scheduler Conflict Check Performance}

Opencast allows to schedule recordings using the Administrative User Interface
(Admin UI) or the External API. When new events are added, the system has to
check for conflicts since two events cannot be recorded at the same time by the
same capture agent.

Due to an architectural flaw, the checks performed by the system are
currently very slow. The more events and capture agents you have, the
more complex and slower this check gets. This leads to a slow,
unresponsive UI on the one hand and can bring the whole system down on
the other hand, in case someone creates many scheduled events within a
short period. This can happen by accident or normal usage and
represents a hidden risk for the availability of your Opencast
installation.

Changing the underlying data structure requires reworking a fair bit of
code. With scalability in mind, a new implementation will speed up the
conflict checking significantly. This will improve user experience and
availability on all sizes of Opencast installations.


\section*{Update Elasticsearch}

Opencast is on
\href{https://www.elastic.co/products/elasticsearch}{Elasticsearch} version
1.7.6. Again, this does not
\href{https://www.elastic.co/support/matrix#matrix_jvm}{work with Java 9} and
is furthermore not supported anymore. This means it will not get any future
security updates or any other maintenance.

The Opencast community has already discussed that upgrading
this is highly desirable and something that needs to be done in the
long run. This not only ensures Java~9 compatibility but also serves as a
backend for an updated player/portal and to replace the old Solr
services. It was also one of the most discussed points in the recent
meetings of Opencast's new high-availability working group.

The search indexes are what currently power (and limit!) all of our
user interfaces. In case of Elasticsearch, its inability to run
distributed currently makes it a single point of failure. That is a
problem when you think about highly available systems. To put it
bluntly: Upgrading Elasticsearch is one of the first steps we need to
do when thinking about removing the admin node as a single point of
failure.

\end{document}
